\usepackage{amssymb}
\usepackage{actuarialangle}
\usepackage{actuarialsymbol}
\usepackage{enumitem}
\everymath{\displaystyle}
\usepackage{enumerate}
\newcommand{\Pro}{\mathbb{P}}
\newcommand{\ent}{\Rightarrow}
\usepackage{mathtools}
\usepackage{amsfonts}
\usepackage{amsthm}
\usepackage{amssymb}
\usepackage{mathrsfs}
\usepackage{enumerate}
\usepackage{subcaption}
\usepackage[spanish]{babel} %% para caracteres en español
% Leonardo añade este paquete para generar la bibliografía con formato APA
\usepackage[natbibapa]{apacite}

\linespread{1.5}
\theoremstyle{remark}
\newtheorem*{Respuesta}{Respuesta}
\usepackage[colorlinks=true,linkcolor=blue,citecolor=black]{hyperref}
\usepackage[a4paper]{geometry}
\usepackage{float}
\geometry{top=2.5 cm, bottom=2.5cm, left=2.5cm, right=2.5cm}
